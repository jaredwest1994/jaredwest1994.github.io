%document class
\documentclass[12pt]{article}

%packages (libraries)

%paper margins
\usepackage[margin=1in]{geometry}

%formatting
\linespread{1.0}

%begin document
\title{Carbonara}
\author{Jared West}

%required to begin the document
\begin{document}
%\maketitle  % -> add to include title information


\begin{center}
    \section*{\huge{Carbonara}}
    \paragraph{}
    A traditional carbonara with a modern twist! (please don't attack me)
\end{center}

\section*{Ingredients}

%ingredient list
\begin{itemize}
    \item 200g pasta (dried or fresh)
    \item 100g guanciale (pork jowl)
    \item 1 clove minced garlic
    \item 50g parmigiano-reggiano
    \item 50g pecorino-romano
    \item 2 eggs
    \item salt \& pepper to taste
\end{itemize}

\paragraph{}
Prepare all ingredients before cooking (mise en place), this is a quick dish so it's best to be prepared! (prep: 5 min, cook: 15 min)
\section*{Process}

\begin{enumerate}
    \item Boil salty water (as salty as the sea) and put chopped guanciale into cold pan and set stove to medium-high.
    \item Add pasta once water begins to boil. (8 min for dried, 90 sec for fresh)
    \item Once guanciale fat is rendered out (remove some if needed - should cover bottom of pan), add fresh cracked pepper, and garlic. Cook until the garlic starts to turn golden.
    \item In a bowl, mix the parm and pecorino cheeses into beaten eggs
    \item After pasta is 80\% cooked, add to guanciale pan, remove from heat and begin mixing around
    \item Add the cheesy egg mixture and stir constantly to ensure a consistent creamy sauce with no egg clumps!
    \item Once creamy sauce is established, plate and enjoy
\end{enumerate}
\end{document}